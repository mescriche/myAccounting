\documentclass[12pt, a4paper]{article}
\usepackage[utf8]{inputenc}
\usepackage[spanish]{babel}
\usepackage{hyperref}
\usepackage{graphicx}
\usepackage{tabularx}
\usepackage{enumitem}
\graphicspath{{./images/}}
\usepackage{tikz}
\usetikzlibrary{babel, positioning, calc}
\usepackage{subfiles}
\title{Mis Apuntes de Contabilidad}
\author{Manuel Escriche Vicente}
\begin{document}
\maketitle
\begin{abstract}
El número de eventos con aspecto económico es vertiginoso; día a día se suceden, aquellos motivados por compras con la tarjeta aparecen en el teléfono móvil o en el reloj; pero hay otros, como los recibos de la electricidad, el agua, o los impuestos, que están silenciosamente registrados en la cuenta corriente. 
 
Mantenerse orientado por el saldo de la cuenta corriente, da cierto grado de control, pero sin perspectiva global. No tienes visión de los flujos entrantes y salientes de dinero, y tampoco tienes perspectiva del capital acumulado.

Emprender algún proyecto económico requiere entender mejor nuestras finanzas personales, nuestra estructura de gastos, nuestra capacidad de ahorro, nuestro flujo de ingresos, nuestro capital acumulado. 

Objetivo: Conseguir una perspectiva global de mis finanzas para establecer objetivos, y realizar proyectos.
\end{abstract}
\tableofcontents
\newpage

\section{Conceptos básicos}
La contabilidad intenta obtener una representación fiable, real de eventos económicos, que se materializan mediante transacciones, que quedan reflejadas en asientos contables.
\par
Los asientos contables son un elemento clave de la actividad contable, y tienen el siguiente aspecto:
\begin{figure}[h]
\centering
 \begin{tikzpicture}
 \draw[help lines] (0,0) grid (10,3);
 \coordinate (a) at (0,2);
 \coordinate (a1) at ($(a) + (1.5, 1)$);
 \coordinate (b)  at ($(a) + (10,0)$);
 \coordinate (b1) at ($(b) + (-1.5, 1)$);
 %\fill[red] (b) circle (2pt);
 %\path[draw, thick] ($(a) + (0,1)$) node[above right] {Transaction \#id} -- ($(b) + (0,1)$) node[above left] {Date};
 \path[draw, ultra thick] (a) node [above right] {Debit}  -- (b) node[above left] {Credit} node[midway,above=2pt]{Accounts};
 \path[draw, ultra thick] (a1) -- +(0,-3) ; 
 \path[draw, ultra thick] (b1) -- +(0,-3);
 \end{tikzpicture}
\end{figure}
%
\paragraph{Account} se refiere a las cuentas que agregan entradas definidas en  el asiento. Estas cuentas pueden ser: por un lado, débito o crédito, y por otro, reales o nominales. Esto configura cuatro tipo de cuentas: real-débito, real-crédito, nominal-débito, nominal-crédito.
\paragraph{Las cuentas reales} son variables de estado, representan el valor de un bien, para las cuentas de débito, y los derechos sobre bienes, para las cuentas de crédito. 
\paragraph{Las cuentas nominales} son variables de flujo, representan o etiquetan la entrada o salida de valor en una cuenta.
\paragraph{''debitar'' y ''acreditar''} las cuentas tienen un significado distinto dependiendo de que la cuenta sea de débito o crédito. De tal forma que:
\begin{itemize}
\item \emph{debitar} una cuenta de débito es \textbf{incrementarla}, en cambio, una cuenta de crédito es \textbf{reducirla}
\item \emph{acreditar} una cuenta de crédito es también \textbf{incrementarla}, en cambio, una cuenta de débito es \textbf{reducirla}
\end{itemize}

\section{Joven Graduado}
\subfile{myAccountingNotes-Graduado.tex}

\clearpage

\section{PYME}
\subfile{myAccountingNotes-PYME.tex}

\end{document}